
\chapter{绪论}
\section{研究目的和意义}
目前,一般民众的上网行为容易被中间人监控、服务端追踪,容易泄漏个人隐私信息,包括政府、运营商、企业,甚至网络窃听者等。
现有的Shadowsocks、V2Ray、Trojan等代理工具被不法分子用来盈利,不符合《网络安全法》,而且通常只有一跳出境。
不少租赁节点使用同一个CDN,可能受到来自CDN的共谋攻击。
而现有的VPN如OpenVPN\cite{feilner2006openvpn}、WireGuard\cite{donenfeld2017wireguard}协议特征明显,能够被精准识别。

因此,本课题研究一种对抗中间人流量分析的安全链路,为用户提供安全不易追踪的上网服务。

\section{国内外研究现状}
\subsection{中间人流量分析技术的演进}
对流量进行归类、分析,是许多应用程序的重要组成部分,例如服务质量 (QoS) 控制、恶意软件检测和入侵检测。
由于其重要性,许多不同的流量分析技术被提出,以适应不同应用场景的需求变化。
随着时间的推移,流量分析技术有着显著的发展和进步。
最近随着通信方面的新进展,包括流量加密和端口混淆,给中间人流量分析带来新的挑战。

第一种也是最简单的方法是探测流量所使用的端口号。但是,这种分析技术的准确性一直在下降,因为较新的应用程序通常使用众所周知的端口号伪装他们的流量或不使用标准注册端口数字。
尽管不准确,但端口号探测仍然广泛在实践中单独使用或与其他流量分析技术一起使用。
下一代流量分析方法,依赖于数据包载荷(payload)或深度数据包检测 (DPI),侧重于发现模式或数据包中的关键字,这些方法只适用到未加密的流量并且具有很高的计算开销。
作为结果,基于流量统计的新一代方法出现了,例如可以通过聚类算法(K-Means和DBSCAN\cite{ester1996density})识别流量\cite{erman2006traffic}。
这些方法依赖于统计或时间序列功能,使他们能够用时处理加密和未加密的流量。
这些方法通常采用经典的机器学习 (ML) 算法,例如随机森林 (RF)和 k最近邻 (KNN)\cite{altman1992introduction}。
然而,这些方法对流量分析的效果在很大程度上取决于人为设计的特征,这限制它们的普遍性。

深度学习不需要领域专家选择特征,因为它会自动选择特征训练。
这一特性使得深度学习成为一种非常理想的流量分析方法,尤其是当新的流量不断涌现,旧流量的模式也在演变。
深度学习的另一个重要特点是它的学习能力显著高于传统的机器学习方法,因此可以学习高度复杂的模式。
结合这两个特点,作为端到端方法,深度学习能够学习非线性原始输入和相应输出之间的关系无需将问题分解为特征选择和分类的小子问题。
最近的工作证明了基于深度学习的流量分析的准确性和高效性,特别是加密的流量\cite{rezaei2019deep}。

\subsection{对抗中间人流量分析的手段}
对数据流量加密是对抗中间人流量分析的通用方法,在加密技术之上,出现了像Shadowsocks,V2Ray,Trojan等代理工具。

Shadowsocks\cite{Shadowsocks}是一种基于SOCKS5代理方式的加密传输协议,也可以指实现这个协议的各种开发包。主要运行原理是对客户端和中转服务器的数据进行加密和混淆。

V2Ray使用VMess协议\cite{vmess}作为客户端和服务器之间的加密通讯协议。VMess的客户端发起一次请求,服务器判断该请求是否来自一个合法的客户端。如验证通过,则转发该请求,并把获得的响应发回给客户端。认证信息是对一个用户ID(一个随机UUID)和时间的16 字节哈希(hash)值。

Trojan\cite{trojan}通过模仿HTTPS流量来实现,当一个客户端连接服务器,会首先进行TLS握手。如果握手正确,之后的数据传输都会被TLS加密,否则服务器会断开与客户端的连接(如同一个真实的HTTPS服务器)。

上面三个常用代理工具通过以机场节点方式提供代理接入,而代理中继节点通常是代理商购买了大量的CDN服务,而CDN厂商并非代理服务商所建,存在不安全因素导致隐私泄露。
CDN厂商可以根据流量的特征和流向可以推断出用户使用了哪种代理工具,访问了哪些网站,因此隐私性和安全性难以得到保证。
因此,我们的目标转向自建安全链路。对加密数据流量的分析通常基于TLS的特征,本课题通过使用传输层的QUIC协议与应用层的WebSocket\cite{rfc6455}和WebTransport协议,构建一条多跳的安全加密隧道。

\subsection{QUIC协议相关工作}
QUIC是一个通用的基于UDP协议的传输层协议,最初由Google设计实现并大规模部署\cite{langley2017quic}。它已经被大多数现代网络浏览器所支持,并提交给IETF进行标准化\cite{rfc9000}。
QUIC提供了TCP的替代方案,将密钥交换和加密带到了传输层。通过提供多路复用和用户空间的拥堵控制,QUIC提供了比HTTP over TCP更强的性能,并解决了HOL阻塞问题。
QUIC对流量分类的意义是双重的。
首先,它强化了一个事实,即未来的网络通信将使用更多的加密技术。也就是说,每个会话的信息将有更大比例被加密。
第二,与HTTP/2\cite{rfc7540}类似,QUIC提供的新特性,例如有效载荷加密、多路复用和并发和服务器推送,增加了流量分类的复杂性。


\section{研究内容}
本文研究一种对抗中间人流量分析攻击的安全链路,该链路有三个组成部分:入口节点、中继节点、出口节点。
\begin{enumerate}
  \item 入口节点(客户端)为本地提供HTTP代理和SOCK5代理服务\cite{rfc1928};
  \item 中继节点包括一个节点池,接收来自客户端的访问流量,并在节点池内部进行多跳转发,最后将流量转发给出口节点池;
  \item 出口节点包括一个节点池,接收来自中继节点的流量,在协议解析后,根据访问内容与目标服务器建立连接。
\end{enumerate}

\section{论文组织结构}
本篇论文的组织结构如下所示:

第一章主要介绍本课题的来源与背景、研究目的及意义。同时,简单介绍本课题的主要研究内容。

第二章主要介绍和讨论课题研究包括的相关技术,如流量加密、流量混淆、所使用的传输协议。

第三章主要介绍本课题实现的对抗中间人流量分析的安全链路,包含总体设计以及具体实现。

第四章主要介绍安全链路使用的三种传输协议(QUIC、WebSocket、WebTransport),并对其各自的优缺点进行比较。