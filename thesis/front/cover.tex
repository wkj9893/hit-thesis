% !Mode:: "TeX:UTF-8"

\hitsetup{
  statesecrets={公开},
  natclassifiedindex={TM301.2},
  intclassifiedindex={62-5},
  ctitleone={一种对抗中间人流量},
  ctitletwo={分析攻击的安全链路},
  ctitlecover={一种对抗中间人流量分析攻击的安全链路},
  ctitle={一种对抗中间人流量分析攻击的安全链路},
  cxueke={工学},
  csubject={信息安全},
  caffil={计算学部},
  cauthor={韦昆杰},
  csupervisor={刘立坤},
  cdate={2022年6月10日},
  cstudentid={1181000420},
  etitle={A secure link against man-in-the-middle traffic analysis attacks},
  exueke={Engineering},
  esubject={Computer Science and Technology},
  eaffil={\emultiline[t]{School of Computer Science \\ and Engineering}},
  eauthor={Wei KunJie},
  esupervisor={Liu LiKun},
  ckeywords={安全链路,代理,QUIC, WebSocket, WebTransport},
  ekeywords={Secure Tunnel, Proxy, QUIC, WebSocket, WebTransport},
}

\begin{cabstract}
目前,一般民众的上网行为容易被中间人监控、服务端追踪,容易泄漏个人隐私信息,包括政府、运营商、企业,甚至网络窃听者等。
现有的Shadowsocks、V2Ray、Trojan等代理工具被不法分子用来盈利,不符合《网络安全法》,而且通常只有一跳出境。
不少租赁节点使用同一个CDN,可能受到来自CDN的共谋攻击。
而现有的VPN如OpenVPN、WireGuard协议特征明显,能够被精准识别。

为了解决上述问题,为用户提供安全不易追踪的上网服务,本课题研究一种对抗中间人流量分析的安全链路,基于模块对整个系统进行划分,随后进一步细化了各个模块的具体设计,并给出各个模块的总体设计和实现。
本文实现的安全链路系统主要包括三个模块:入口节点、中继节点和出口节点。
入口节点处在整个加密安全链路系统的起始部分,运行在用户本地,负责作为HTTP代理或者SOCKS5代理将本地流量通过加密隧道传输到目的服务器。
中继节点处在整个加密安全链路系统的中心部分,负责中继用户流量到加密隧道的下一个节点,通常一条完整的加密隧道可以包含多个中继节点。
出口节点处于整个加密安全链路系统的边缘部分,负责接收来自中继节点的连接,并与目标服务器通信。

本文设计实现的基于Go语言的安全链路系统,通过在传输链路设置多跳,每一跳都通过安全传输协议(QUIC、WebSocket、WebTransport)加密传输数据,保证了安全链路的匿名性和安全性。
系统支持二进制代码的跨平台软件开发,同时通过Docker实现了对应用程序的打包,方便部署和使用。  
\end{cabstract}

\begin{eabstract}
At present, the Internet behavior of ordinary people is easily monitored by middlemen and tracked by servers, and it is easy to leak personal privacy information, including the government, operators, enterprises, and even network eavesdroppers.
Existing proxy tools such as Shadowsocks, V2Ray, and Trojan are used by criminals for profit, do not comply with the Cybersecurity Law, and usually only have one jump out of the country.
Many rental nodes use the same CDN and may be subject to collusion attacks from the CDN.
Existing VPNs such as OpenVPN and WireGuard have obvious features and can be accurately identified.
  
In order to solve the above problems and provide users with safe and untraceable Internet services, this topic studies a secure link against middleman traffic analysis, divides the entire system based on modules, and then further refines the specific design of each module, and gives Out the overall design and implementation of each module.
The secure link system implemented in this paper mainly includes three modules: entry node, relay node and exit node.
The entry node is at the beginning of the entire encrypted secure link system, runs locally on the user, and acts as an HTTP proxy or SOCKS5 proxy to transmit local traffic to the destination server through an encrypted tunnel.
The relay node is located in the central part of the entire encrypted secure link system, and is responsible for relaying user traffic to the next node of the encrypted tunnel. Usually, a complete encrypted tunnel can include multiple relay nodes.
The exit node is at the edge of the entire encrypted secure link system and is responsible for receiving connections from relay nodes and communicating with the target server.
  
The security link system based on Go language designed and implemented in this paper, by setting multiple hops in the transmission link, each hop encrypts the transmission data through the security transmission protocol (QUIC, WebSocket, WebTransport), which ensures the anonymity and the security of the link. safety.
The system supports cross-platform software development of binary code, and at the same time realizes the packaging of applications through Docker, which is convenient for deployment and use.
\end{eabstract}
