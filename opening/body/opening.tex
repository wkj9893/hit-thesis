
\section{课题来源及研究的目的和意义}
\subsection{课题来源}
本课题来源于深度数据包检测(DPI)中的字符串匹配部分,主要研究在压缩流量场景下的高性能字符串匹配算法,可以应用于DPI系统提高其对压缩流量的匹配效率。

\vspace{3mm}
\subsection{研究背景与意义}
深度数据包检测(Deep packet inspection,缩写为DPI),是一种计算机网络数据包过滤技术,通过对数据包的分析然后采取警报、阻塞、重定向、日志记录等相应操作。字符串匹配算法是DPI的核心组成部分,高效的字符串匹配算法能够提高整个DPI系统的运行效率。

如今,服务器通常会压缩要传输的数据,来减少传输量和延时。通过gzip算法\cite{RFC1952},文本数据(HTML、CSS和JavaScript)的压缩率可以达到25\%,这意味着理论上我们对压缩数据的匹配效率最高可以达到未压缩数据的4倍。

本课题实现在压缩流量场景下的高性能字符串匹配算法,根据要匹配的模式(字符串或正则表达式),对压缩数据可以实现快速匹配。另外,本课题实现的系统可以应用于DPI系统检测压缩数据。
\vspace{8mm}
\section{国内外在该方向的研究现状及分析}
\subsection{总体情况}
字符串匹配算法自20世纪就得到广泛的研究,从单模式匹配下的KMP算法\cite{knuth1977fast}、BM算法\cite{boyer1977fast},到多模式匹配下的AC算法\cite{aho1975efficient},在到Commentz-Walter算法\cite{commentz1979string}、Rabin–Karp算法\cite{karp1987efficient}。

针对压缩流量的字符串匹配算法(ACCH算法)最早在2011年被提出\cite{bremler2011accelerating},随后针对压缩流量的正则表达式匹配算法(ARCH算法)\cite{becchi2015accelerating}也在2015年被提出。然后,2020年twins算法\cite{sun2020efficient}被提出,进一步提高了对压缩流量匹配的效率。

目前,针对压缩流量的字符串匹配算法中都基于AC算法,我们可以对AC算法作出改进来进一步提高算法的运行效率\cite{李雪莹2004字符串匹配技术研究}。并且,我们可以选用其他多模式匹配算法,在特定场景下代替AC算法。

\vspace{3mm}
\subsection{gzip算法}
gzip算法\cite{RFC1952}是现在压缩HTTP流量最常用的算法,超过90\%的Alexa前五百名网站都使用gzip算法作为默认压缩算法。gzip算法基于DEFLATE算法,首先通过LZ77算法对内容进行压缩,然后再进行哈夫曼编码(Huffman coding)。LZ77算法扫描时,记录首次出现的字符串位置,将后面与该字符串相同的字符串记录为二元组(length,distance),其中length表示字符串的长度,distance表示当前字符串与最早相同字符串的距离。例如,字符串abcdefabcd可以被压缩为abcdef(6,4)。假设一个字符串的长度是10字节,那么这个字符串以2字节的二元组表示,压缩率可以达到1/5。

\vspace{3mm}
\subsection{朴素算法}
对压缩流量最简单的匹配方法:每次遇到压缩数据的二元组,根据长度和距离确定其所指向的原字符串,然后对原字符串进行匹配。假设匹配压缩数据的长度为$n$,压缩率是$m\%$,如果使用基于DFA的AC算法,匹配时间为$n/m$。

\vspace{3mm}
\subsection{AC算法}
AC算法是最知名的多模式匹配算法,被广泛应用于snort等DPI系统,原论文\cite{aho1975efficient}提出两种方法,一种是通过构建goto函数、failure函数、output函数,可以看作构建一个NFA。然后根据字符的输入进行状态的转移,如果goto函数值不为0进入下一个状态并向前移动,否则根据fail函数只进行状态的转移而不移动。如果状态刚好处于output函数,匹配成功相应的模式串。由于文本总长度为n,每输入一个字符,要么根据goto函数进入新状态并且向前移动,要么根据failure函数进入与当前状态后缀的最长前缀的状态并且不移动,由于状态0必定会发生移动,因此总的匹配时间小于$2n$。假设所有模式串的总长度为$m$,要搜索的文本长度为$n$,状态节点的数量小于$m$,总的空间复杂度小于$3m$

除了这种方式外,还可以构建DFA,思路是将goto函数和failure函数用一个二维数组代替(full matrix representation),二维数组保存了所有状态接收一个字符进入的新状态。假设二维数组为arr,二维数组的行数是所有的状态数量,列数通常是256(一个字节),arr[i][j]=k表示状态i接收j对应单个字节的输入进入状态k。根据DFA构建的AC算法从状态0开始,每接收一个字符进入一个新的状态并向前移动,因此总的匹配时间是$n$。然而由于这种方法没有进行压缩,总的空间复杂度为$256m$,所需内存空间较NFA更大,实际情况往往采用第一种方法。

\vspace{3mm}
\subsection{Twins算法}
Twins算法借助gzip压缩的信息实现对压缩数据的高效匹配。AC的DFA是确定性的,给定当前状态s和输入c,下一个状态被唯一确定。因此,对压缩数据的二元组,如果在某一位置状态与对应的第一个字符串相同,后面的状态也将完全相同,就可以直接跳过这些状态转移。例如,对于abcdef<6,4>,如果我们要匹配的模式是abcd,那么由于二元组之前的状态和初始状态相同,因此我们可以直接跳过最后的二元组。

\vspace{8mm}
\section{主要研究内容}

本课题主要研究内容是实现在压缩流量场景下的高性能字符串匹配算法,可以应用于DPI系统检测压缩数据。主要的研究内容包括以下3个部分:

\subsection{数据收集}
数据收集环节,本课题选择Alexa.com和Alexa.cn排名前500的网站页面作为要收集的原始数据。本课题采用网络爬虫的方式,通过主动方式获取网站数据。根据网站的URL,本模块构造相应的请求报文并从服务器获取相应报文,收集一定数量的数据,并将全部数据压缩保存起来。

\vspace{3mm}
\subsection{数据处理}
数据处理环节主要是对snort规则数据的处理,方便进行测试。snort\cite{roesch1999snort}是一套开放源代码的网络入侵防御系统(IPS),能够做到实时流量分析和数据包记录,被认为是全世界最广泛使用的入侵预防与侦测软件。snort规则基于正则表达式,本课题选择不同数量的snort规则作为原始数据,数据处理环节需要根据snort规则构建相应的DFA。

\vspace{3mm}
\subsection{性能比较}
性能比较环节包括对实现的各种字符串匹配算法的运行效率和占用内存的比较,主要步骤是在相同数据集下,比较不同算法的运行时间和占用内存。最后,对于实现的最优算法,还将比较其在单核和多核下的性能。

\vspace{8mm}
\section{研究方案}
选取Alexa.com和Alexa.cn前500名的网站作为原始流量数据,选取不同数量的snort规则(snort24,snort31,snort34,snort135)作为原始数据。实现已经提出的不同压缩流量字符串匹配算法,并在相同压缩数据场景下进行比较。然后选取测试结果最好的算法,改进其中的AC算法,得到效率更高的算法。

\vspace{8mm}
\section{进度安排,预期达到的目标}
2021.11-2021.12:查询各种压缩流量下的字符串匹配算法的原论文并进行实现,在测试数据下测试其正确性

2022.1-2022.2:对实现的程序在收集到的压缩数据下进行测试,比较运行效率和占用内存

2022.3:根据比较的结果,改进算法,实现压缩流量场景下的高性能算法,并在收集的流量数据下与原匹配算法进行比较

2022.4:根据最终结果,完成毕业论文

\vspace{8mm}
\section{课题已具备和所需的条件、经费}
系统平台:Linux

硬件环境:
\begin{itemize}
  \item CPU:Intel i5-8265U (8) @ 3.900GHz
  \item GPU:NVIDIA GeForce MX250    
  \item 内存:8GB
  \item 硬盘容量:512GB
\end{itemize}

编程语言:C++

软件环境:Visual Studio Code

\vspace{8mm}
\section{研究过程中可能遇到的困难和问题,解决的措施}
\begin{enumerate}
  \item 算法改进:\\对字符串匹配算法的优化、压缩没有太深的了解。在实现字符串匹配算法的同时进一步学习,同时查阅各算法的原论文,掌握其思想。
  \item 数据处理:\\snort规则基于正则表达式,因此需要首先转化为DFA才能进行匹配。许多库函数实现了正则表达式转化为DFA,可以直接调用或者自己根据正则表达式转化为DFA的方法\cite{张树壮2011面向网络安全的正则表达式匹配技术}实现。
  \item 编程实现:\\本课题计划通过C++实现算法,CMake作为组建自动化工具,目前来讲对于整体的程序设计和程序测试上缺乏经验,需要一段时间去熟悉。
\end{enumerate}

\vspace{8mm}
\section{主要参考文献}

\bibliographystyle{hithesis}
\bibliography{references}
